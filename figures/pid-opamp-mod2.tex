\documentclass[]{standalone}
\usepackage[utf8]{inputenc}
\usepackage[american]{circuitikz}
\usetikzlibrary{arrows,shapes,calc,positioning}

\begin{document}

\pgfmathsetmacro\opdisth{7}
\pgfmathsetmacro\opdistv{6}

\begin{circuitikz}[scale=1]
\node[op amp] (op1) {\bf C};
\node[op amp] (op2) at ($ (op1.+) + (7,0) $) {\bf I};
\node[op amp] (op3) at ($ (op2) + (0,\opdistv{}) $) {\bf P};
\node[op amp] (op4) at ($ (op2) - (0, \opdistv{}) $) {\bf D};
%\node[op amp, xscale=-1] (op5) at ($ (op2.+) - (-3,5) $) {};
\node[op amp] (op5) at ($ (op2.+) + (\opdisth{},0) $) {};
\node at (op5) {\bf S};

\node[coordinate] (feedback) at ($ (op4.-) + (-9,0) $) {};
\node[coordinate] (refinput) at ($ (op1.+) + (-2,0) $) {};
\node[coordinate] (out) at ($ (op5.out) + (2,0) $) {};
\node[coordinate] (sum) at ($ (op5.-) + (-1,0) $) {};
\node[coordinate] (dist) at ($ (sum) + (0,3) $) {};
\node[coordinate] (pot) at ($ (sum) + (0,5) $) {};

% Connect to ground
\foreach\i in {1,5}{%
    \draw (op\i.+) to[R=$R_{0}$,*-] ++(0,-1.5) node[ground] {};
}
% Connect to ground
\foreach\i in {2,3,4}{%
    \draw (op\i.+) to ++(0,-1.2) node[ground] {};
}

% Connect op-amps
\foreach \i/\n/\r in {1/2/I, 1/3/0, 2/5/0,  3/5/0}{%
    \draw (op\i.out) to [R=$R_{\r}$] (op\n.-);
}

\draw (op4.out) to [R=$2R_0$] (op5.+);

% Capacitans at input to 4 (Derivator)
%    \draw (op6.out) to [C=$C_{D}$] (op4.-);

% Feedback resistors in op-amp 1, 3,5,6
\foreach \i/\r in {1/0, 3/P ,5/0}{%
    \draw (op\i.-) to ++(0,1.3) to[R=$R_{\r}$] ++(2.2,0) -| (op\i.out);
}

% Feedback capacitor in op-amp 2 (integrator)
\foreach\i in {2}{%
    \draw (op\i.-) to ++(0,1.6) to[C=$C_I$] ++(2.2,0) -| (op\i.out);
}

% Feedback circuit in op-amp 4 (differentiator) R and C in parallell
\draw (op4.-) to ++(0,1.5) to ++(0.2, 0) to ++ (0, 0.4) to[R=$R_D$] ++(2.0, 0) to ++(0,-0.4) 
to node[coordinate, pos=1] (connect) {} ++(0.2, 0) to (op4.out);
\draw (connect) to ++(-0.2, 0) to ++ (0, -0.4) to[C=$C_2$] ++(-2.0, 0) to ++(0,0.4) 
to ++(-0.2, 0) to (op4.-);

% \draw (op5.out) to ++(0,1.6) to[R=$R_{2}$] ++(2.2,0) -| (op5.-);

% Input
\draw (feedback) node[below] {$y(t)$} to[short, o-] (feedback |- op1.-)  to[R=$R_{0}$, ] (op1.-);
\draw (feedback)  to[C=$C_{D}$, o-] (op4.-);
\draw (op1.+) to[R=$R_{0}$, -o, ] node[below] {$y_{ref}(t)$}  (refinput);

% Output
\draw (op5.out) to[short, *-o] (out) node[below] {$u(t)$};

% Feedback
%\draw (op3.out) to[short,*-]  (op3.out |- op5.-) to [R=$R_{1}$,-*] (op5.-);
%\draw (op4.out) to ++(0,-8) to [R=$R_{1}$] ++(-8,0) -| (sum);

% Sum feedback signals
%\draw (op5.out) to[R=$R_1$] ++(-2,0) -| (sum);

% Potentiometer
% Disturbance input
%\node[ground] at (pot) {};
%\draw  (pot) to[american potentiometer,n=mypot,-o] ++(0,2) 
%  (mypot.wiper) to ++(0,-2) node[left] {disturbance}  to[R=$R_1$,]  ++(0,-2) -| (sum);

%\draw (sum) to (dist) to[R=$R_1$, -o] ++(-2,0) node[below] {$v(t)$};

\end{circuitikz}

\end{document}