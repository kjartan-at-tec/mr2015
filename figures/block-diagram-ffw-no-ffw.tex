\documentclass[]{standalone}
\usepackage[utf8]{inputenc}
\usepackage[american]{circuitikz}
\usetikzlibrary{arrows,shapes,calc,positioning}

\usepackage{graphicx}


\begin{document}
 
  \begin{tikzpicture}[node distance=22mm, block/.style={rectangle, draw, minimum width=15mm, minimum height=10mm}, sumnode/.style={circle, draw, inner sep=2pt}]
    
    \node[coordinate] (input) {};
    \node[sumnode, right of=input, node distance=16mm] (sum2) {\tiny $\Sigma$};
    \node[block, right of=sum2, node distance=20mm] (controller2)  {$F(s)$};
    %\node[sumnode, right of=controller2, node distance=20mm] (sumffw) {\tiny $\Sigma$};
    \node[coordinate, right of=controller2, node distance=20mm] (sumffw) {\tiny $\Sigma$};
    %\node[block, above of=sumffw, node distance=22mm] (Fv) {$F_v(s)$};
    \node[block, right of=sumffw, node distance=30mm] (plant1)  {$G(s)$};
    \node[sumnode, right of=plant1, node distance=20mm] (sumdist) {\tiny $\Sigma$};
    \node[coordinate, right of=sumdist, node distance=25mm] (output) {};
    \node[block, above of=sumdist, node distance=22mm] (Hv) {$H_v(s)$};
    \node[coordinate, above of=Hv, node distance=20mm] (dist) {};

    \draw[->] (input) -- node[above, pos=0.3] {$r(t)$} (sum2);
    \draw[->] (sum2) -- node[above] {$e(t)$} (controller2);
    \draw[->] (controller2) -- node[above] {$u(t)$} (plant1);
    \draw[->] (sumdist) -- node[coordinate,] (m2) {} node[above, near end] {$y(t)$} (output);
    \draw[->] (m2) -- ++(0,-15mm) -| node[right, pos=0.95] {-} (sum2);


    \draw[->] (dist) -- node[near start, right] {$v(t)$} node[coordinate] (mv) {} (Hv);
    \draw[->] (Hv) -- node[near start, right] {} (sumdist);
    \draw[->] (plant1) -- node[near start, right] {} (sumdist);

    
  \end{tikzpicture}

  
\end{document}
