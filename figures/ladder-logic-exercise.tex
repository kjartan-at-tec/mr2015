\documentclass[]{standalone}
\usepackage[utf8]{inputenc}
\usepackage[american]{circuitikz}
\usetikzlibrary{arrows,shapes,calc,positioning}
\usetikzlibrary{circuits.plc.ladder}

\usepackage{graphicx}


%\newcommand*{\coil}[1]{to[short] ++(0.5, 0) node[coordinate] (orig) {} arc [start angle=180, end angle=150,radius=8mm] (orig) arc [start angle=180, end angle=210,radius=8mm] (orig) ++(1cm, 0) node[coordinate] (coilend) {} arc [start angle=0, end angle=30,radius=8mm] (coilend) arc [start angle=0, end angle=-30,radius=8mm] (coilend) to[short] ++(0.5cm, 0) (orig) ++(0.5, 0.8) node {#1}}

\begin{document}

\small


\pgfmathsetmacro\rightrail{8}
\pgfmathsetmacro\rungone{-1}
\pgfmathsetmacro\pdist{1.5}
\pgfmathsetmacro\rungtwo{-4}

\begin{tikzpicture}[circuit plc ladder, scale=1.5]

  % Rails
  \draw (0,0) to[short, o-]  (0,-7);
  \draw (\rightrail,0) to[short, o-] (\rightrail,-7);

  % First rung
  \draw (0, \rungone) to[contact NO={info={``Aext''}, info'={I0.2}},]  (2, \rungone) to[contact NC={info={PB\\ ``Reset''}, info'={I0.1}},]
  (3.5, \rungone) to[short] (6,\rungone) to[coil={info={``Pressed''},info'={M0.0}},] (\rightrail, \rungone);
  \draw (0, \rungone -\pdist) to[contact NO={info={``Pressed''}, info'={M0.0}},] (2, \rungone - \pdist) to[short,] (2, \rungone);

  % Second rung
  \draw (0, \rungtwo) to[contact NO={info={PB ``Start''}, info'={I0.0}},]  (2, \rungtwo) to[contact NC={info={``Pressed''},info'={M 0.0}},]
  (4, \rungtwo) to[contact NC={info={``Aext''}, info'={I0.2}},] (6,\rungtwo) to[coil={info={``Valve''}, info'={Q0.0}},] (\rightrail, \rungtwo);
  \draw (0, \rungtwo -\pdist) to[contact NO={info={``Valve''}, info'={Q0.0}}] (4, \rungtwo - \pdist) to[short,] (4, \rungtwo);


  
\end{tikzpicture}

\end{document}
