\documentclass[]{standalone}
\usepackage[utf8]{inputenc}
\usepackage[american]{circuitikz}
\usetikzlibrary{arrows,shapes,calc,positioning}

\begin{document}

\pgfmathsetmacro\circuitheight{6}
\pgfmathsetmacro\circuitwidth{7}

\begin{circuitikz}[scale=1]
  \begin{scope}
    \draw
    (0,0) node[and port] (myand) {}
    (myand.in 1) node[anchor=east] {A}
    (myand.in 2) node[anchor=east] {B}
    (myand.out) node[anchor=west] {AB};
  \end{scope}
  \begin{scope}[yshift=-7cm]
    \node[coordinate] (Aplus) at (1.5, \circuitheight) {};
    \node[coordinate] (Amin) at (1.5, 0) {};
    \node[coordinate] (Bplus) at (3, \circuitheight) {};
    \node[coordinate] (Bmin) at (3, 0) {};
    \draw (0,0) to[american voltage source, label=5V] (0, \circuitheight) 
    to[short,-o] ++(\circuitwidth,0);
    \draw (0,0) to[short, -o] (\circuitwidth,0);
    \node[and port] (and1) at (6, 4)  {};
    \draw (Aplus) to[switch=A, -*] (Aplus |- and1.in 1) node[coordinate] (Ac) {} to[R=$R_1$] (Amin);
    \draw (Bplus) to[switch=B, -*] (Bplus |- and1.in 2)  node[coordinate] (Bc) {} to[R=$R_1$] (Bmin);
    \draw (Ac) to[short,] (and1.in 1);
    \draw (Bc) to[short,] (and1.in 2);
    \draw (and1.out) to[R=$R_2$,] ++(0,-1.6) to[empty led,] (and1.out |- 0,0);
    \draw (0,0) to[short,] (0,-0.5) node[ground] {};
  \end{scope}
\end{circuitikz}
\end{document}
